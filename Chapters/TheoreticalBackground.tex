\chapter{Theoretical Background}
\label{chap:theoretical_background}

This chapter outlines the conceptual and scientific foundations relevant to the development of an XR-based assistance system for industrial workers. It introduces key technologies, principles of ergonomics, human-machine interaction modalities, and architectural tools that enable real-time support in complex environments.

\section{Extended Reality in Industrial Contexts}
\subsection{Definition and Taxonomy}

\subsection{Applications of XR in Industry}

\subsection{Challenges and Opportunities}

\section{Human Factors and Ergonomics}
\subsection{Principles of Ergonomic Design}

\subsection{Common Industrial Posture-Related Injuries}

\subsection{Need for Real-Time Monitoring}

\section{Wearable Technologies for Assistance}
\subsection{Overview of Industrial Wearables}

\subsection{The Teslasuit}

\subsection{Capabilities and Limitations}

\section{Haptic Feedback and Multimodal Warning Systems}
\subsection{Perception of Haptic Stimuli}

\subsection{Design Principles for Effective Feedback}

\subsection{Visual vs. Haptic Alerts}

\subsection{Cognitive Load and Safety Feedback}

\section{Spatial Awareness and Localization in XR}
\subsection{XR Tracking Technologies}

\subsection{The Meta Quest 3}

\subsection{Spatial Consistency with Teslasuit Data}

\section{Multisensor Integration and Synchronization}
\subsection{Challenges in Multimodal Systems}

\subsection{LabStreamingLayer (LSL)}

\subsection{Integration of MQTT and Environment Data}

\section{Related Work}

