\chapter{Introduction}
In modern industrial environments, ensuring the safety and well-being of workers is
an increasingly complex challenge. This is due to the rise in automation of factories,
which has led to workers being exposed to a range of risks from moving parts to robots.
Traditional safety measures (e.g., warning signs, training programs) are often static and
thus insufficient for modern dynamic workspaces \cite{robb2019vrtraining}. They do not
adapt to the real-time context of the worker, leading to potential oversights.

Consider a typical workstation on an assembly line: a robotic arm picks components
from a moving conveyor and places them onto a workbench, where a human operator
performs detailed finishing tasks. The process is fast-paced, and both the worker and
the robot operate in close proximity. A moment of inattention could result in the worker
reaching into the robot’s operational area, or adopting a strained posture while handling a
part, creating risks of both acute injury and long-term musculoskeletal strain \cite{punnett2004wrmsd}.
Traditional safety would not be able to provide the necessary real-time feedback to prevent such
incidents since they rely on the worker’s ability to notice and process external warnings
while engaged in their primary tasks.

To address these shortcomings, a new approach that leverages real-time data and im-
mersive technologies can provide more effective assistance. By integrating real-time
physiological and spatial data with immersive and haptic feedback, it is possible to create
a system that enhances worker safety in dynamic environments. Such a system can make
risks visible, posture perceptible, and hazards tangible, thereby supporting workers in
making safer decisions and maintaining ergonomic practices.

Advances in extended reality (XR) and wearable technology, such as full-body haptic
suits and immersive head-mounted displays, are opening new possibilities for real-time
assistance and feedback systems. These technologies are no longer confined to research
labs or entertainment industries but are being increasingly considered for applications
in healthcare, rehabilitation, training, and industrial safety \cite{palmarini2018ar}. This increased availability of
XR and wearable devices presents the opportunity to develop such adaptive assistance
systems that can provide real-time, personalized feedback to workers.

Current approaches at such systems often focus on one mode of feedback, or do not
provide real-time, personalized assistance, resulting in them not utilizing the full potential
of XR and wearable technologies. Additionally, systems utilizing only a singular modality
(e.g., only visual feedback) can lead to information overload through excessive visual
stimuli, which can distract workers from their primary tasks. By combining multiple
modalities—such as visual, auditory, and haptic feedback—these systems can provide
more intuitive and effective warnings that are less likely to be ignored or seem distractive
\cite{lindeman2004haptic}.


This thesis presents the design and development of an XR-based assistance framework for industrial workers that integrates real-time physiological and spatial data from wearable sensors with immersive visual and haptic feedback. The system is tailored for dynamic, collaborative work environments, such as assembly lines where humans and robots operate in close proximity, and aims to enhance situational awareness, promote ergonomic posture, and reduce the risk of accidents without disrupting workflow.

