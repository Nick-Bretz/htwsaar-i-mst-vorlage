\chapter{Introduction}
In modern industrial environments, ensuring the safety and well-being of workers is an increasingly complex challenge. This is due to the rise in automation of factories, which has led to workers being exposed to a range of risks from moving parts to robots. Traditional safety measures like warning signs, floor markings, and standard training provide a necessary foundation, but they are not always sufficient. In many cases, these static systems fail to reach workers at the right moment, especially when attention is focused elsewhere or hazards change rapidly. These systems require real-time adaptation and individualization.

To protect workers more effectively, safety solutions must become adaptive, personalized, and accessible. They must understand the individual context of the worker: how they move, where they are in the environment, and when they may be at risk, whether from poor posture, fatigue, or close proximity to danger zones.

Advances in extended reality (XR) and wearable technology, such as full-body haptic suits and immersive head-mounted displays, are opening new avenues for real-time assistance and feedback systems. These technologies are no longer confined to research labs or entertainment industries but are being increasingly considered for applications in healthcare, rehabilitation, training, and industrial safety. In this context, integrating real-time physiological and spatial data with immersive and haptic feedback presents a promising opportunity to enhance worker safety in hazardous zones. 

Based on the identified challenges and objectives, this thesis investigates the following research questions:

\begin{enumerate}
  \item \textbf{RQ1:} How effectively can physiological and postural data from the Teslasuit be used to detect unsafe posture in real time?
  \item \textbf{RQ2:} What latency thresholds are acceptable for haptic and visual feedback in XR-based industrial safety systems?
  \item \textbf{RQ3:} How can spatial data from XR devices be aligned with sensor data from wearables to create a reliable danger zone alert system?

This thesis explores how such a system could be designed and deployed. Specifically, it proposes an XR-based assistance framework that helps workers stay safe by making risk visible, posture perceptible, and hazards tangible. By integrating real-time motion, biometric, and positional data, the system aims to provide immediate, personalized support without disrupting the workflow or overwhelming the user. Rather than replacing human judgment, it enhances situational awareness and supports safer, more ergonomic behavior in industrial settings.