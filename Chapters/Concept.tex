\chapter{Concept}
\label{chap:concept}

\section{Introduction}
% Recap of the research problem and objectives of this chapter.
This chapter presents the conceptual design of an XR-based assistance system aimed at enhancing the safety and ergonomics of industrial workers. Building on the theoretical foundations outlined in Chapter \ref{chap:theoretical_background}, it details the system requirements, architectural framework, posture analysis and hazard detection concepts, feedback modalities, interaction design, and evaluation criteria. The goal is to provide a comprehensive blueprint that addresses the identified challenges while leveraging the capabilities of wearable technologies and XR environments. This conceptual framework serves as a foundation for the subsequent implementation and evaluation phases of the project.

\section{System Requirements and Constraints}

\subsection{Functional Requirements}
-Real-time posture monitoring
-Hazard detection in XR environment
-Haptic and visual feedback for unsafe postures and hazards
-Data logging for ergonomic assessment
-User-friendly interface for workers
-Integration with existing industrial systems (GAIA)

\subsection{Non-functional Requirements}
-Latency 
-Intrusiveness
-Reliability
-Usability
\subsection{Hardware and Software Constraints}
-control center only runs on windows so component running windows is required for the teslasuit
-teslasuit not that precise and cant detect whether body part rested on something or not

\section{Framework Architecture}
-due to the software limitations the Framework is distributed between the Meta Quest 3 and a windows PC which will handle the data and most of the computation
\subsection{Hardware Components}
-Meta Quest 3
-Teslasuit
-Windows PC
-GAIA system
-potential addtional sensors (smart watch etc)
\subsection{Data Flow and Integration via LSL}
\subsection{Communication Protocols}

\section{Posture Analysis Concept}
-Real-time joint angle data acquisition from Teslasuit
-Ergonomic assessment using RULA
\subsection{Joint Angle Data Acquisition}
\subsection{Ergonomic Assessment Model}
\subsection{Mapping to Ergonomic Risk Levels}

\section{Hazard Detection Concept}
-Using XR environment to define and visualize danger zones
-Integration of real-world data for dynamic hazard detection
\subsection{Spatial Alignment of XR and Real-World Coordinates}
-
\subsection{Definition of Danger Zones}
\subsection{Sensor and XR Data Fusion}

\section{Feedback Modalities}
\subsection{Haptic Feedback}
\subsection{Visual Feedback}
\subsection{Rationale for Feedback Choice}

\section{Interaction Design}
-Aside from setup and calibration the system should require minimal interaction from the worker
-The system should provide clear and concise feedback without overwhelming the worker
- The system should be non-intrusive when no hazards or poor posture is detected
\subsection{Worker Interaction with the System}

\subsection{Avoiding Information Overload}
-Prioritization of Alerts
-Haptic vs Visual Feedback
-low intensity haptic feedback for posture correction
-stronger haptic feedback for hazard alerts
-visual feedback for immediate/ dangerous hazard alerts
\subsection{Safety-Critical Design Considerations}


\section{Conceptual Evaluation Criteria}
\subsection{Relation to Research Questions}
\subsection{Success Criteria for the Concept}

\section{Summary and Transition}
% Recap of the conceptual design and transition to Implementation.
