\chapter{Implementation}

\section{Overview}
This chapter describes the practical realization of the XR-based assistance framework introduced in Chapter~4. It details the integration of hardware and software components, the data processing pipeline, posture analysis, hazard detection, and feedback mechanisms. The implementation emphasizes real-time performance, modularity, and synchronization via the LabStreamingLayer (LSL).

\section{System Setup and Environment}
\subsection{Hardware Configuration}
\begin{itemize}
    \item \textbf{Teslasuit:} Full-body haptic and motion capture suit for physiological and kinematic data.
    \item \textbf{Meta Quest 3:} XR headset used for spatial localization and visual feedback.
    \item \textbf{Host PC:} Windows-based system running the Teslasuit controller and LSL data server.
    \item \textbf{Networking:} Local Wi-Fi connection.
\end{itemize}

\subsection{Software Components}
\begin{itemize}
    \item Unity XR environment with LSL plugin.
    \item Teslasuit Unity SDK and API for motion and haptic control.
    \item Python-based data logging and processing layer.
    \item MQTT broker (optional) for integration with environmental or GAIA data.
    \item RULA algorithm implementation for ergonomic assessment.
    \item Custom Unity scripts for data visualization and user interaction.
\end{itemize}

\subsection{Development Environment}
\begin{itemize}
    \item Unity Version, SDK versions, Python version, and libraries.
    \item Communication architecture diagram (LSL stream definitions, clock synchronization, etc.).
    \item Version control and collaboration tools used (e.g., Git).
\end{itemize}

\section{Data Integration and Synchronization}
\subsection{LabStreamingLayer Implementation}
Describe:
\begin{itemize}
    \item Stream definitions for Teslasuit (motion, biometrics) and Quest (position, rotation).
    \item Stream naming and channel mapping.
    \item Time-synchronization strategy and data rate management.
\end{itemize}

\subsection{Data Logging and Processing Pipeline}
\begin{itemize}
    \item Python receiver for stream verification.
    \item Logging format (CSV, JSON, etc.).
    \item Handling of missing or delayed samples.
\end{itemize}

\section{Posture Analysis Module}
\subsection{Joint Angle Extraction}
\begin{itemize}
    \item Calculation of joint angles/orientations from Teslasuit data needed for RULA.
\end{itemize}

\subsection{RULA-Based Ergonomic Assessment}
The RULA algorithm was implemented directly in Unity via C# scripts. It uses data provided by the Teslasuit Plugin transfered to a skeleton model in Unity. As the skeleton doesn't provide joint angle data directly, these angles were calculated based on the rotation data of the relevant bones. The script processes the required joint angles (e.g., projecting the upper arm angle onto a plane defined by the shoulder orientation) to compute the RULA score in real-time. 
\begin{itemize}
    \item Simplified RULA implementation (without load/support).
    \item Scoring logic (angle thresholds $\rightarrow$ risk level).
    \item Real-time computation and temporal smoothing.
    \item Add pictures of the skeleton model with relevant angles highlighted.
\end{itemize}

\subsection{Output Integration}
\begin{itemize}
    \item Visualization of current posture risk in XR (optional UI overlay).
\end{itemize}

\section{Hazard Detection Module}
\subsection{Spatial Calibration}
\begin{itemize}
    \item Coordinate alignment between Quest 3 and Teslasuit.
    \item Calibration routine (e.g., reference pose or fiducial marker alignment).
\end{itemize}

\subsection{Danger Zone Definition}
\begin{itemize}
    \item Zone setup in Unity.
    \item Trigger colliders representing hazardous areas.
\end{itemize}

\subsection{Collision Detection and Feedback Trigger}
\begin{itemize}
    \item Data fusion between spatial tracking and posture module.
    \item Threshold-based activation logic.
\end{itemize}

\section{Feedback System}
\subsection{Haptic Feedback Implementation}
\begin{itemize}
    \item Mapping of body zones to Teslasuit actuators.
    \item EMS patterns for posture vs. hazard alerts.
    \item Latency testing setup and results (link to evaluation).
\end{itemize}

\subsection{Visual Feedback Implementation}
\begin{itemize}
    \item Unity overlay design (color-coded warnings, directional cues).
\end{itemize}

\subsection{Feedback Coordination}
\begin{itemize}
    \item Priority handling to avoid overload.
\end{itemize}

\section{System Evaluation Tools}
\begin{itemize}
    \item Logging framework for latency and accuracy testing.
    \item User study scripts and task definitions (if included).
    \item Collected metrics: latency, risk detection accuracy, user comfort.
\end{itemize}

\section{Summary}
Brief summary connecting the implementation back to research questions:
\begin{itemize}
    \item How physiological data enables posture detection.
    \item How haptic and visual feedback operate together.
    \item How LSL ensures synchronization and low latency.
\end{itemize}
