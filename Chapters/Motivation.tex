\chapter{Motivation}
\label{chap:motivation}
Modern industrial environments are increasingly characterized by high levels of automation, dynamic workflows, and close interaction between human workers and complex machinery. While these developments have improved efficiency, they have also introduced new safety challenges. Workers are required to perform physically demanding tasks in proximity to moving parts, robotic systems, and other potential hazards. In such conditions, it is hard for workers to maintain constant situational awareness, leading to an increased risk of accidents and injuries.

Returning to the typical workstation on an assembly line, an XR based assistance system utilizing information about it's environment  
Traditional safety measures—such as fixed signage, floor markings, or general training programs provide an essential baseline but are inherently limited. They rely on the worker’s ability to notice and process external warnings while engaged in their primary tasks. This static approach does not account for the worker’s individual posture, or the evolving state of their environment. As a result, critical risks may go unnoticed until it is too late.

There is a clear need for safety systems that are adaptive, personalized, and context-aware. Such systems must continuously monitor the worker’s physical state and spatial position, detect unsafe situations as they emerge, and deliver warnings through intuitive, non-disruptive channels. Advances in Extended Reality (XR) and wearable technologies offer a unique opportunity to meet these requirements. Head-mounted displays can overlay hazard information directly into the user’s field of view, while full-body haptic suits can provide spatially localized alerts that are instantly understood without diverting visual attention.

The XR-based assistance framework proposed in this thesis addresses these needs by combining immersive visualization with real-time haptic feedback, powered by continuous analysis of physiological and positional data. Unlike existing solutions, it adapts its feedback to the worker’s specific context, enabling timely interventions without distracting the worker. In doing so, it has the potential to reduce the risk of acute injuries, prevent the development of musculoskeletal disorders, and improve overall situational awareness in industrial environments.

The urgency for such a solution is amplified by current industry trends: increasing automation, a growing focus on occupational health, and the wider availability of high-performance, affordable XR and wearable devices. Now is the right time to explore how these technologies can be integrated into practical safety systems that protect workers without compromising efficiency.
